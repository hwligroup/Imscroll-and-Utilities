\documentclass[a4paper,fleqn,10pt,openright,oldfontcommands, article]{memoir}    
%\documentclass[10pt,fleqn]{letter}    
\usepackage{amsmath}
%\usepackage[minionint,textosf,mathlf]{MinionPro}
\usepackage[english]{babel}
\usepackage{graphicx}

% pagesetup
\pagestyle{plain}

% symbols
\renewcommand{\L}{\ensuremath{\mathcal{L}}}
\newcommand{\pd}{\partial}
\newcommand{\dg}{\dagger}
\newcommand{\T}{\top}

% math notation
\newcommand{\pdff}[2]{\frac{\partial #1}{\partial #2}}
\newcommand{\fdff}[2]{\frac{\delta #1}{\delta #2}}
\newcommand{\ifrac}[2]{#1 / #2}
\newcommand{\abs}[1]{\left\vert #1 \right\vert}
\newcommand{\ten}[1]{\underline{\underline{#1}}}
\newcommand{\p}[2]{\ensuremath{p(#1 \:|\: #2)}}
 \renewcommand{\d}[1]{\ensuremath{\!\!d#1 \:}}

% condensed +/-
\newcommand{\mm}{\ensuremath{\!-\!}}
\newcommand{\pp}{\ensuremath{\!+\!}}
\newcommand{\ee}{\ensuremath{\!=\!}}

% lengths
\setlength{\mathindent}{18pt}
\setlength{\parindent}{0pt}
\setlength{\parskip}{6pt}

% section headings
%\chapterstyle{article}
%\setsecnumdepth{subsection}
%\setcounter{maxsecnumdepth}{0}
%\setcounter{minsecnumdepth}{1}
%\setcounter{maxsecnumdepth}{1}

%\setsecnumformat{\llap{\csname the#1\endcsname\hspace{\baselineskip}}}

\setsecheadstyle{\large\bfseries}
% \setsecheadstyle{\sscshape\raggedright\MakeLowercase}
\setbeforesecskip{5pt plus 2pt minus 2pt}
\setaftersecskip{5pt plus 2pt minus 2pt}

\newcommand{\ps}[2]{\ensuremath{p^*(#1 \:|\: #2)}}
\newcommand{\q}[2]{\ensuremath{q(#1 \:|\: #2)}}

\begin{document}

\chapter*{Hierarchical Inference on Single-molecule Time Series,\\ using VBEM and Emperical Bayes on HMM's} 

\begin{KeepFromToc}
  \tableofcontents
\end{KeepFromToc}

\chapter{Model Properties}

\section{Data, latent states, parameters}

\hspace{\mathindent}
\begin{tabular}{ll}
  $x = \{x_n\} = \{\{x_{n,t}\}\}$   %
    & Observation in trace $n \in \{1 \ldots N\}$ at time $t \in \{1 \ldots T_n\}$ \\
  $z = \{z_n\} = \{\{z_{n,t}\}\}$ %
     & State of molecule $n$ at time $t$ \\
  \smallskip \\  
  $\theta = \{\theta_n\} = \{\pi_n, A_n, \mu_n, \lambda_n\}$ %      
    & Parameters for trace $n$\\
  \smallskip \\  
  $\pi_n = \{\pi_{n,k}\}$ %
    & Initial probabilities: Prob that trace starts in state $k$ \\
  $A_n = \{\{A_{n,k,l}\}\}$ %
    & Transition matrix: Prob of moving from state $k$ to state $l$\\
  $\mu_n = \{\mu_{n,k}\}$ %
    & FRET level for state $k$ in trace $n$ \\
  $\lambda_n = \{\lambda_{n,k}\}$ %
    & FRET emissions precision (1/var) for state $k$ in trace $n$ \\
  $u = \{\{u^\mu_k, u^\beta_k, u^W_k, u^\nu_k \}, \{u^A_k\}, \{u^\pi\} \}$ %
    & Hyperparameters for ensemble distribution (prior)
\end{tabular}
\medskip

\section{Evidence}
\begin{align}
  \begin{split}
  \p{x}{u} 
       &= \int \d{\theta} \p{x, \theta}{u} \\
       &= \int \d{\theta} \p{x}{\theta} \p{\theta}{u} \\ 
       &= \int \d{\theta} \prod_n \p{x_n}{\theta_n} \p{\theta_n}{u} \\ 
       &= \prod_n  \int \d{\theta_n} \p{x_n}{\theta_n} \p{\theta_n}{u}
  \end{split}  
\end{align}

\subsection{Likelihood}
\begin{align}
  \begin{split}
  \p{x}{\theta} &= \prod_n \p{x_n}{\theta_n} \\
                &= \prod_n \sum_{z_n} \p{x_n , z_n}{\theta_n} \\
                &= \prod_n \sum_{z_n} \p{x_n}{z_n, \theta_n} \p{z_n}{\theta_n}
  \end{split}
\end{align}

\section{Emissions model}
\begin{align}
  \begin{split}
  \p{x_n}{z_n, \theta_n} &= \prod_t 
                            \p{x_{n,t}}{z_{n,t} \,,\, \theta_{n}} \\
                         &= \prod_t \prod_k 
                            \p{x_{n,t}}{\theta_{n,k}}^{z_{n,t,k}}
  \end{split}\\
  \begin{split}
    \p{x_{n,t}}{\theta_{n,k}} 
    &= N(x_{n,t} \,|\, \mu_{n,k}, \lambda_{n,k}) \\ 
    &= (\lambda_{n,k}/2\pi)^{1/2} \exp[-\tfrac{1}{2} \Delta_{n,t,k}^2] 
  \end{split} \\
  \Delta_{n,t,k}^2
  &= 
  \lambda_{n,k} (x_{n,t} - \mu_{n,k})^2
\end{align}

\section{Transition probabilities (HMM)}
\begin{align}
 \p{z_n}{\theta_n} &= \left[
                        \prod_{t=2}^{T_n} \p{z_{n,t}}{z_{n,t-1}, \theta_n}
                      \right]
                      \p{z_{n,1}}{\theta_n} \\
  \p{z_{n,t}}{z_{n,t-1} \,,\, \theta_{n}}
      &= \prod_{k,l} (A_{n,k,l})^{z_{n,t-1,k} \, z_{n,t,l}} \\
  \p{z_{1}}{\theta_{n}} 
      &= \prod_k (\pi_{n,k})^{z_{n,1,k}}
\end{align}

\section{Ensemble Distributions (VBEM Priors)}
\begin{align}
 \begin{split}
 \p{\theta_n}{u}
    &= \p{\pi_n}{u} \p{A_n}{u} \p{\mu_n, \lambda_n}{u} \\
    &= \p{\pi_n}{u} \prod_k \p{A_{n,k}}{u} \p{\mu_{n,k}, \lambda_{n,k}}{u} 
 \end{split}\\
 \pi_n 
    &\sim \mathrm{Dir}(u^{\pi}) \\
 A_{n,k} 
    &\sim \mathrm{Dir}(u^{A}_{k}) \\
 \lambda_{n,k}
    &\sim \mathrm{Wish}(u^{W}_k, u^{\nu}_k) \\
 \mu_{n,k}
    &\sim \mathrm{N}(u^{\mu}_k, u^{\beta}_k \lambda_{n,k}) 
\end{align}

\section{Algorithm Outline}

Loop over iterations $i$ until $\sum_n \L_n$ converges:
\begin{itemize}
 \item[1.] VBEM updates: obtain $q^{(i)}(\theta_n)$, $q^{(i)}(z_n)$, and $\L^{(i)}_n$ for each trace $n$, using $p^{(i-1)}(\theta_n)$ as the VBEM prior on the paramters.
 \item[2.] Hierarchical updates: solve for 
  \[
    p^{(i)}(\theta \:|\: u) = \arg\max_{p(\theta)} \sum_n \L^{(i)}_n[q^{(i)}(z_n), q^{(i)}(\theta_n), \p{\theta}{u}]
  \]
\end{itemize}

\chapter{Conjugate-Exponential Form}

Given that all our likelihoods and prior are in the exponential family, the likelihood $\p{x}{\eta}$ and the prior $\p{\eta}{\nu, \chi}$ can be written in a common general form:
\begin{align}
  \p{x}{\eta} &= f(x) g(\theta) \exp[\eta \cdot \upsilon(x)] \\
  \p{\eta}{\nu, \chi} &= h(\nu, \chi) g(\theta)^\nu \exp [\eta \cdot \chi]
\end{align}
where $\eta$ represents the remapped parameters $\theta$, and $\{\nu, \chi\}$ represent the remapped hyperparameters to this general form. The posterior $\p{\eta}{x,\nu,\chi}$ now takes the form:
\begin{align}
  \p{\eta}{x,\nu,\chi} 
  &\propto 
  \p{x}{\eta}\p{\eta}{\nu,\chi} \\
  &=
  f(x) h(\nu, \chi) g^{\nu + 1} \exp [ \eta \cdot (\chi + \upsilon(x)) ]
\end{align}
which of course yields a distribution of the same form as the prior
\begin{align}
  \p{\eta}{x,\nu,\chi} 
  &=
  \p{\eta}{\tilde\nu, \tilde\chi}
\end{align}
with parameters
\begin{align}
  \tilde \nu
  &= 
  \nu + 1 \\
  \tilde \chi
  &=
  \chi + \upsilon(x)
\end{align}
In generalized exponential form, the hyperparameter $\nu$ can be interpreted as scale factor, that encodes the number of  spreviously observed samples. The hyperparameter vector $\chi$, in turn takes the role of the summed sufficient statistics $\upsilon(x)$ associated with each of the samples. 

\section{Normal-Gamma}

\begin{align}
  &
  \p{x}{\mu,\lambda} 
  =
  N(x \:|\: \mu, \lambda)\\
  &
  \p{\mu, \lambda}{u^\mu, u^\beta, u^a, u^b}
  =
  N(\mu \:|\: u^\mu, \lambda u^\beta) \mathrm{Gamma}(\lambda \:|\: u^a, u^b) 
\end{align}
\begin{align}
  \eta &= \{ \lambda, \lambda \mu \} \\
  \nu &= u^\beta = 2 u^a - 1 \\
  \chi &= \{ -\tfrac{1}{2}(u^\beta (u^\mu)^2 + 2 u^b), u^\beta u^\mu \}
\end{align}
\begin{align}
  g(\eta) &= (\eta_1 / 2\pi)^{1/2} \exp [-\eta_2^2 / 2 \eta_1] \\
  \upsilon(x) &= \{-\tfrac{1}{2} x^2, x \}
\end{align}
\begin{align}
  f(x) &= 1 \\
  h(\nu, \chi) &= (2 \pi)^{(\nu-1)/2}  \nu^{1/2}  (-\chi_1 - \chi_2^2/\nu)^{(\nu+1)/2}
\end{align}

\section{Normal-Wishart (1d)} \label{ssec:conj-exp-norm-wish}

\begin{align}
  &
  \p{x}{\mu,\lambda} 
  =
  N(x \:|\: \mu, \lambda)\\
  &
  \p{\mu, \lambda}{u^\mu, u^\beta, u^a, u^b}
  =
  N(\mu \:|\: u^\mu, \lambda u^\beta) \mathrm{Wish}(\lambda \:|\: u^W, u^\nu) 
\end{align}
\begin{align}
  \eta &= \{ \lambda, \lambda \mu \} \\
  \nu &= u^\beta = u^\nu - 1 \\
  \chi &= \{ -\tfrac{1}{2}(u^\beta (u^\mu)^2 + 1/u^W), u^\beta u^\mu \}
\end{align}
Functions $g(\eta)$, $u(x)$, $f(x)$ and $h(\nu, \chi)$ are the same as with a Normal-Gamma distribution.

\section{Dirichlet} \label{ssec:conj-exp-dir}

\begin{align}
  &
  \p{z}{\pi} 
  =
  \mathrm{Cat}(z \:|\: \pi) = \prod_k \pi_k ^ {z_k}\\
  &
  \p{\pi}{u^\pi}
  =
  \mathrm{Dir}(\pi \:|\: u^\pi)
  = \frac{\Gamma(\sum_k u^\pi_k)}{\prod_k \Gamma(u^\pi_k)} \prod_k \pi_k ^ {u^\pi_k - 1}  
\end{align}
\begin{align}
  \eta &= \{ \ln \pi_k \} \\
  \nu &= 1 \\
  \chi &= \{ u^\pi_k \}
\end{align}
\begin{align}
  g(\eta) &= 1 \\
  \upsilon(z) &= \{ z_k \}
\end{align}
\begin{align}
  f(x) &= 1 \\
  h(\nu, \chi) &= \frac{\prod_k \Gamma(\chi_k+1)}{\Gamma(\sum_k (\chi_k + 1))}
\end{align}


\chapter{Variational Bayes Expectation Maximization (VBEM)}

\emph{Note}: We will omit the $n$-subscript in this section, since VBEM is performed on one trace at a time.

When performing (structured) VBEM on a Hidden Markov Model, we introduce an approximating factorization for the posterior $\p{z,\theta}{x} \simeq q(z)q(\theta)$, that allows calculation of a lower bound on the log-evidence (using Jensen's inequality):
\begin{align}
  \begin{split}
  \ln p(x) &= \ln
              \left[
                \int \d{\theta}
                \sum_{z} 
                p(x, z, \theta) 
              \right] \\
           &= \ln
              \left[
                \int \d{\theta}
                \sum_{z}
                  q(z) q(\theta) 
                  \frac{p(x, z, \theta)}
                       {q(z) q(\theta)}
              \right] \\
           &\ge \int \d{\theta}
                \sum_{z}  
                q(z) q(\theta)
                \ln
                \left[
                  \frac{p(x, z, \theta)}
                       {q(z) q(\theta)}  
                \right] \\
           &= \L[q(z), q(\theta)]
  \end{split}
\end{align} 
The lower bound $\L$ is tight if $q(z)q(\theta) = \p{z, \theta}{x}$:
\begin{align}
  \begin{split}
  \L[q(z), q(\theta)]
    &=
    \int \d{\theta}
    \sum_{z}  
    q(z) q(\theta)
    \ln
    \left[
      \frac{p(x, z, \theta)}
            {q(z) q(\theta)}  
    \right] \\  
    &=
    \int \d{\theta}
    \sum_{z}  
    \p{z, \theta}{x}
    \ln
    \left[
      \frac{p(x, z, \theta)}
            {\p{z, \theta}{x}}  
    \right] \\  
    &=
    \int \d{\theta}
    \sum_{z}  
    \p{z, \theta}{x}
    \ln
    \left[
      \frac{\p{z, \theta}{x} p(x)}
           {\p{z, \theta}{x}}  
    \right] \\  
    &=
    \int \d{\theta}
    \sum_{z}  
    \p{z, \theta}{x}
    \ln p(x) \\  
    &=
    \ln p(x)
    \int \d{\theta}
    \sum_{z}  
    \p{z, \theta}{x} \\
    &=
     \ln p(x)
  \end{split}
\end{align}

\section{Updates}

Loop until $\L$ converges. For $i$-th iteration:
\begin{itemize}
 \item[1.] E-step: keeping $q^{(i)}(\theta)$ fixed, solve for
   \[
     q^{(i+1)}(z) = \arg\max_{q(z)} \L[q(z), q^{(i)}(\theta)]
   \]
   
 \item[2.] M-step: keeping $q^{(i)}(z)$ fixed, solve for  
   \[
     q^{(i+1)}(\theta) = \arg\max_{q(\theta)} \L[q^{(i)}(z), q(\theta)]
   \]
   
\end{itemize}


\section{E-step}

To maximize $\L$ w.r.t. $q(z)$, we solve $\nabla_{q(z)} \L = 0$, introducing a Lagrange multiplier $\lambda_z$ to ensure normalization:
\begin{align}
  \begin{split}
  0 = \nabla_{q(z)} &
  \left[ 
    \L[q(z), q(\theta)] 
    + \lambda_z \left( 1 - \sum_{z'} q(z') \right)
  \right] \\
  = &
  \left[
  \int \d{\theta} 
  q(\theta)
  \left(
  \ln p(x, z, \theta)
  - \ln q(z) 
  - \ln q(\theta)
  - 1
  \right)
  \right]
  - \lambda_z
  \end{split}
\end{align}
We can pull $\ln q(z)$ out of the integral, since it has no dependence on $\theta$. This yields
\begin{align}
  \begin{split}
  \ln q(z) 
  &=
    \left[
    \int \d{\theta}  
    q(\theta)
    \left(
    \ln p(x, z, \theta)
    - \ln q(\theta) 
    - 1
    \right)
    \right]
    - \lambda_z \\
  &=
    E_{q(\theta)} [ \ln p(x, z, \theta) ] -  
    E_{q(\theta)} [ \ln q(\theta) ] - (1 + \lambda_{\theta})\\
  &=
    E_{q(\theta)} [ \ln p(x, z, \theta) ] -  
    \ln Z_{q(\theta)}
  \end{split}
\end{align}
here we have absorbed all terms without a $z$-dependence into a constant $\ln Z_{q(z)}$. The approximate posterior $q(z)$ is obtained by taking the exponent of the above equation
\begin{align}
  q(z) &= \frac{1}{Z_{q(\theta)}} 
            \exp \left[ E_{q(\theta)} [ \ln p(x, z, \theta) ] \right]
\end{align}
where normalization of $q(z)$ implies
\begin{align}
  Z_{q(z)}
  &= \sum_{z} \exp \left[ E_{q(\theta)} [ \ln p(x, z, \theta) ] \right]
\end{align}

The expectation of $p(x, z, \theta)$ w.r.t. $q(\theta)$ expands to:
\begin{align}
  E_{q(\theta)} [ \ln p(x, z, \theta) ]
  & =
  \int \d{\theta} 
  q(\theta)
  \left[
  \ln \p{x}{z, \theta} 
  + \ln \p{z}{\theta}
  + \ln \p{\theta}{u}
  \right]
\end{align}
The $z$-dependent terms can be written as:
\begin{align}
  \begin{split}
    E_{q(\theta)} [ \ln \p{x}{z, \theta} ]
    = &  
    \sum_t \sum_k 
    z_{t,k} 
    \int \d{\theta} q(\theta)
    \left[
      \tfrac{1}{2} \ln \left( \ifrac{\lambda_{k}}{2\pi} \right)
      - \tfrac{1}{2} \Delta^2
    \right] \\
    = &  
    \sum_t  
    z_{t}^{\T} 
    \cdot 
    E_{q(\theta)} [ \tfrac{1}{2} \ln \left( \ifrac{\lambda_{:}}{2\pi} \right)
                            - \tfrac{1}{2} \Delta^2]
  \end{split}
\end{align}
and
\begin{align}
  \begin{split}
    E_{q(\theta)} [ \ln \p{z}{\theta} ] 
    = & 
    \sum_{t=2}^{T}
    \sum_{k,l}
    z_{t,l} z_{t \mm 1,k} 
    \int \d{\theta} q(\theta)
    \ln A_{kl} \\
    & +
    \sum_k
    z_{1,k}
    \int \d{\theta} q(\theta)
    \ln \pi_{k} \\
    = & 
    \sum_{t=2}^{T}
    z_{t - 1}^{\T} 
    \cdot
    E_{q(\theta)} [ \ln A ]
    \cdot 
    z_{t} 
    + z_{t}^{\T} 
    \cdot 
    E_{q(\theta)} [\ln \pi] 
    % %
    %   \ln \p{\theta}{u}
    %   = & \\
  \end{split}
\end{align}
Note that we do not need the expectation of the prior $E_{q(\theta)}[p(\theta)]$, since
\begin{align}
  \label{eq:qz}
  \begin{split}
    q(z) 
    = & 
    \frac{\exp \left( E_{q(\theta)}[\ln p(x, z, \theta)] \right)}
          {\sum_{z} \exp \left(\ln E_{q(\theta)}[p(x, z, \theta)] \right)} \\
    = &
    \frac{\exp 
          \left( 
            E_{q(\theta)}[\ln \p{x, z}{\theta}] 
            + E_{q(\theta)}[\ln p(\theta)] 
          \right)}
          {\sum_{z} 
          \exp 
          \left( 
            E_{q(\theta)}[\ln \p{x, z}{\theta}] 
            + E_{q(\theta)}[\ln p(\theta)] 
          \right)} \\
    = &
    \frac{\exp 
          \left( 
            E_{q(\theta)}[\ln \p{x, z}{\theta}] 
          \right)}
          {\sum_{z} 
          \exp 
          \left( 
            E_{q(\theta)}[\ln \p{x, z}{\theta}] 
          \right)}
    \frac{\exp 
          \left( 
            E_{q(\theta)}[\ln p(\theta)] 
          \right)}
          { 
          \exp 
          \left( 
            E_{q(\theta)}[\ln p(\theta)] 
          \right)}\\
    = &
    \frac{\exp 
          \left( 
            E_{q(\theta)}[\ln \p{x, z}{\theta}] 
          \right)}
          {\sum_{z} 
          \exp 
          \left( 
            E_{q(\theta)}[\ln \p{x, z}{\theta}] 
          \right)}
  \end{split}
\end{align}
We see that the posterior $q(z)$ is parametrized by expectation under $q(\theta)$ of the squared Mahalanobis distance $E_{q(\theta)}[\Delta_{t,k}^2]$, and the logarithm of the parameters $E_{q(\theta)}[\ln \lambda]$, $E_{q(\theta)}[\ln A]$ and $E_{q(\theta)}[\ln \pi]$. This allows us to define
\begin{align}
  \label{eq:q_z_star}
  \begin{split}
    q(z)
    &= \frac{1}{\hat{Z}_{q(z)}} p^*(x, z)\\
    p^*(x, z) 
    &= \exp \left[ E_{q(\theta)}[\ln \p{x, z}{\theta}] \right] \\
    \hat{Z}_{q(z)}
    &= \sum_{z} p^*(x, z)
    = p^*(x) = Z_{q(z)} / E[p(\theta)]_{q(\theta)}
  \end{split}
\end{align}
which decomposes into
\begin{align}
  p^*(x, z) 
  &= \left[ \prod_t p^*(x_{t} \,|\, z_{t}) \right] p^*(z \,|\, \theta) \\
  p^*(x_{t} \,|\, z_{t}=k) 
  &= (\lambda^*_{k}/2 \pi)^{1/2} \exp \left[-\tfrac{1}{2} \Delta_{t,k}^{*2} \right]\\
  p^*(z \,|\, \theta) &= p(z \,|\, A^*, \pi^*) 
\end{align}
by defining
\begin{align}
  \Delta^{*2} &= E_{q(\theta)} [ \Delta^2] \\
  \ln \lambda^* &= E_{q(\theta)} [ \ln \lambda ] \\
  \ln A^* &= E_{q(\theta)} [ \ln A ] \\
  \ln \pi^* &= E_{q(\theta)} [ \ln \pi ]
\end{align}
This result is a specific example of a general property of all exponential family models with conjugate likelihood/prior pairs: we can always find a set of point-estimates $\eta^*$ such that
(\emph{reference Beal here})
\begin{align}
 q(z) &= \frac{1}{Z_{q(\eta)}} \exp[E_{q(\eta)} [\ln p(x, z, \eta) ]] = \frac{1}{Z_{q(\eta)}} p(x, z, \eta^*)
\end{align}
In our specific case, this result implies that we could in principle compute some $\eta^*$ for the natural parameters for the Normal-Wishart distribution $\eta = \{ \lambda, \lambda \mu \}$, such that $\p{x}{\eta^*_k} = (\lambda^*_k/2 \pi)^{1/2} \exp[-\tfrac{1}{2}\Delta^{*2}]$. However for the purposes of implementing the VBEM algorithm, this step is not required to calculate $q(z)$.

From the analytical forms of the priors, we can express the point estimates as (\emph{TODO: verify the algebra here}):
\begin{align}
  \Delta^{*2} &= 
  (1/w^\beta_k)
  + 
  w^\nu_k w^W_k (x - w^\mu_k)^2 \\
  \ln \lambda^* &= \psi (w^\nu_k) + \ln 2 w^W_k\\
  \ln A_{k,l}^* &= \psi \left( w^A_{k,l} \right) - \psi \left( \sum_l w^A_{k,l} \right)\\
  \ln \pi_{k}^* &= \psi \left( w^\pi_{k} \right) - \psi \left( \sum_k w^\pi_{k} \right)
\end{align}

In practice, we do not calculate $q(z)$ for all $K^T$ possible paths through the state space (which would be numerically unfeasible). Rather, show in the next section that the updates for $q(\theta)$ only require knowledge of a set of point estimates of the state $z_{t,k}$ and transition correlation $z_{t-1,k}z_{t,l}$. We will show how to calculate these using the \emph{forward-backward} algorithm at the end of the section.

\section{M-step} 

In the m-step we maximize $\L$ w.r.t. $q(\theta)$. Again $\lambda_\theta$ is a Lagrange multiplier. We now take the functional derivative instead of a gradient, but the steps are essentially the same.
\begin{align}
  0 = \fdff{}{q(\theta)} &
  \left[ 
    \L[q(z), q(\theta)] 
    + \lambda_{\theta} \left( 1 - \int \d{\theta'} q(\theta') \right)
  \right] \\
  = & 
  \left[
    \sum_{z}  
    q(z)
    \left(
    \ln p(x, z, \theta)
    - \ln q(z) 
    - \ln q(\theta)
    - 1
    \right)
  \right]
  - \lambda_\theta
\end{align}
like in the E-step, this reduces to
\begin{align}
  \ln q(\theta) 
  &=
    \left[
    \sum_{z}  
    q(z)
    \left(
    \ln p(x, z, \theta)
    - \ln q(z) 
    - 1
    \right)
    \right]
    - \lambda_\theta \\
  &=
    E_{q(z)} [ \ln p(x, z, \theta) ] -  
    E_{q(z)} [ \ln q(z) ] - (1 + \lambda_\theta)\\
  &=
    E_{q(z)} [ \ln p(x, z, \theta) ] -  
    \ln Z_{q(\theta)} 
\end{align}
with normalization constant $Z_{q(\theta)}$
\begin{align}
  \ln Z_{q(\theta)}
  &= \int \d{\theta} \sum_{z} q(z) \ln p(x, z, \theta)
\end{align}
The expectation of $\ln p(x, z, \theta)$ expands to:
\begin{align}
  \begin{split}
    E_{q(z)}[\ln p(x, z, \theta)]
    =&
    E_{q(z)} [ \ln \p{x}{z, \theta} 
    + E_{q(z)} [ \ln \p{z}{\theta} ] \\
    &+ \ln \p{\theta}{u}
  \end{split}
\end{align}
where the $z$-dependent terms become:
\begin{align}
  \begin{split}
    E_{q(z)} [ \ln \p{x}{z, \theta} ]
    = &  
    \sum_{t} \sum_{k}  
    E_{q(z)} [z_{t,k}] 
    \left[ 
      \tfrac{1}{2} \ln \left( \ifrac{\lambda_{k}}{2\pi} \right)
      - \tfrac{1}{2} \Delta_{t,k}^2
    \right]
  \end{split} \\
  \begin{split}
    E_{q(z)} [ \ln \p{z}{\theta} ] 
    = & 
    \sum_{t=2}^{T}
    \sum_{k,l}
    E_{q(z)} [z_{t,l} z_{t \mm 1,k}] 
    \ln A_{kl} \\
    & + 
    \sum_k
    E_{q(z)} [z_{1,k}]
    \ln \pi_{k} 
  \end{split}
\end{align}
the sufficient statistics for $q(z)$, which can be calculated with a forward backward algorithm (see below), are given by:
\begin{align}
 \gamma_{t,k} &= E_{q(z)} [z_{t,k}] \\
 \xi_{t,kl} &= E_{q(z)} [z_{t-1,k} z_{t,l}]
\end{align}
and the expression for $q(\theta)$ can be rewritten as:
\begin{align}
  \label{eq:q_theta}
  \begin{split}
    q(\theta)
    =& 
    \frac{p(\theta | u)}{Z_{q(\theta)}} 
    \prod_{t,k}
    \left(
      \left( 
      \ifrac{\lambda_{k}}{2\pi} \right)^{1/2}
      \exp 
      \left[
          - \tfrac{1}{2} \Delta_{t,k}^2
      \right]
    \right)^{\gamma_{t,k}}\\
    &
    \prod_{t=2,k,l}
    \left(
      A_{kl}
    \right)^{\xi_{t,kl}} 
    \prod_{k}
    \left(
      \pi_k
    \right)^{\gamma_{1,k}}
  \end{split}
\end{align}
Again we see that we can write:
\begin{align}
  p^*(x, z, \theta) &= \exp \left[ \ln E_{q(z)}[p(x, z, \theta)] \right]
\end{align}
where the integral over $q(z)$ can be expressed through the substitutions
\begin{align*}
 z^*_{t,k} &= \gamma_{t,k} \\ 
 (z_{t-1,k} z_{t,l})^* &= \xi_{t,k,l}
\end{align*}
Note also that the following decomposition for $q(\theta)$ holds without further need for approximation:
\begin{align}
  q(\theta) 
  &= q(\mu, \lambda) q(A) q(\pi)
\end{align}
This in turn means we can write:
\begin{align}
  \begin{split}
    \label{eq:q_mu_l}
    q(\mu, \lambda) 
    =& \ps{x}{z, \mu_{n}, \lambda_{n}} p(\mu, \lambda) \\
    =& \prod_k \left[\prod_t \p{x_{t}}{\mu_{k}, \lambda_{k}}^{\gamma_{t,k}} \right] 
        p(\mu_{k}, \lambda_{k}) 
  \end{split}\\
  \label{eq:q_A}
  q(A) &= \ps{z_{2:T}}{z_{1}, A} p(A) \\
  \label{eq:q_pi}
  q(\pi) &= \ps{z_{1}}{\pi}
\end{align}
This means that the m-step reduces to calculation of a set of \emph{variational} parameters $w$ that determines $q(\theta | w)$ from the \emph{hyperparameters} $u$ that define $p(\theta|u)$ and the sufficient statistics for $\ps{x,z}{\theta}$.

In order to calculated the updates for $\q{\mu,\lambda}{w}$ we will use the fact that the prior and likelihood are exponential family, so that they may be written as:
\begin{align}
  \begin{split}
    \q{\eta}{\tilde{\nu}_{n}, \tilde{\chi}_{n}}
    = &
    h \left(\tilde{\nu_{n}}, \tilde{\chi_{n}} \right) 
    g(\eta)^{\tilde{\nu}_{n}} 
    \exp \left[ \eta \cdot \tilde{\chi}_{n} \right] \\
    = &
    Z_{q(\theta)}^{-1}
    f(x) g(\eta)
    \exp \left[ \eta \cdot \upsilon(x_{n}) \right] \\
    & h \left(\nu_{n}, \chi_{n} \right) 
    g(\eta)^{\nu_{n}} 
    \exp \left[ \eta \cdot \chi_{n}  \right] \\
    = &
    \tilde{Z}_{q(\theta)}^{-1}
    g(\eta)^{\nu_{n} + 1} 
    \exp \left[ \eta \cdot (\chi_{n} + \upsilon(x_{n})) \right] 
  \end{split}
\end{align}
This allows us to rewrite equation (\ref{eq:q_mu_l}) as:
\begin{align}
  \q{\eta_{k}}{\tilde{\nu}_{k}, \tilde{\chi}_{k}}
  =& \tilde{Z}_{q(\theta)}^{-1} 
      \left[
        \prod_t
        \left(
          g(\eta_{k}) \exp \left[ \eta_{k} \cdot \upsilon(x_{t}) \right]
        \right)^{\gamma_{t,k}}
      \right] \\
     & g(\eta_{k}) \exp \left[ \eta_{k} \cdot \chi_{k} \right]
\end{align}
which yields the updates
\begin{align}
  \tilde{\nu}_{k} &= \nu_{k} + \sum_t \gamma_{t,k} \\
  \tilde{\chi}_{k} &= \chi_{k} + \sum_t \gamma_{t,k} \upsilon(x_{t})
\end{align}
 We can now substitute 
\begin{align}
  \nu &= u^\beta = u^\nu - 1 \\
  \chi &= \{ -\tfrac{1}{2}(\nu (u^\mu)^2 + 1/u^W) \,,\, \nu u^\mu \} \\
  \upsilon(x) &= \{-\tfrac{1}{2} x^2, x\} 
\end{align}
and define
\begin{align}
  N_{k} &= \sum_t \gamma_{t,k} \\
  \bar{X}_{k} &= \sum_t \gamma_{t,k} x_{t} \\
  \bar{X^2}_{k} &= \sum_t \gamma_{t,k} x_{t}^2 
\end{align}
to obtain the following expressions for the variational parameters $\q{\theta}{w}$:
\begin{align}
  w^\mu_{k} =& \tilde{\chi}_{k,2} / \tilde{\nu}_{k} 
         = (u^\beta_k u^\mu_k + \bar{X}_{k})/(u^\beta_k + N_{k}) \\
  w^\beta_{k} =& u^{\beta}_k + N_{k} \\
  w^\nu_{k} =& u^{\nu}_k + N_{k} \\
  \begin{split}
    (w^W_{k})^{-1} 
    =& 
      - \tilde{\chi}_{k,2} 
      - \tfrac{1}{2} \tilde{\chi}_{k,2}^2/\tilde{\nu}_{k} \\
    =&
    \left(
      (u^{W}_k)^{-1} + u^\beta_k (u^\mu_k)^2 + \bar{X^2}_{k}
    \right) \\
    &- \frac{1}{2}
      \left(
      (u^\beta_k u^\mu_k + \bar{X}_{k})^2
      /(u^\beta_k + N_{k})
      \right)
  \end{split}
\end{align}
Finally, the updates for $u^A$ and $u^\pi$ can be obtained by substitution of the terms in equation (\ref{eq:q_theta}) into equations (\ref{eq:q_A}) and (\ref{eq:q_pi}):
\begin{align}
  w^A_{kl} &= u^A_{kl} + \sum_{t=2}^{T} \xi_{t,kl} \\
  w^\pi_{k} &= u^\pi_{k} + \gamma_{1,k}
\end{align}

We now proceed to derive how $\gamma$ and $\xi$ can be calculated using the Forward-backward algorithm.

\section{Forward-Backward Algorithm}

The forward-backward algorithm is a method to calculate expectation values under the posterior $p(z | x, \theta)$, or in our case, the approximate posterior $q(z)$ of a Hidden Markov Model:  
\begin{align}
 \gamma_{t,k} 
  &= E_{q(z)} [z_{t,k}] 
   = \ps{x_1}{z_1} p^*(z_1) \\
 \xi_{t,kl} 
  &= E_{q(z)} [z_{t-1,k} z_{t,l}]
   = \ps{z_{t-1}=k, z_{t-1}=l}{x_{1:T}} 
\end{align}
to do so we calculate two variables:
\begin{align}
 \alpha_{t,k} &= p^*(x_{1:t}, z_t=k) \\
 \beta_{t,k} &= \ps{z_t=k}{x_{t+1:T}}
\end{align}
such that
\begin{align}
  \gamma_{t,k} &= \ps{z_t=k}{x_{1:T}}
                = \frac{\alpha_{t,k} \beta_{t,k}}{p^*(x_{1:T})} \\
  \xi_{t,k,l} &= \ps{z_{t-1}=k, z_{t-1}=l}{x_{1:T}} \\
              &= \frac{\ps{x_{1:T}}{z_t, z_{t-1}} p^*(z_t, z_{t-1})}{p^*(x_{1:T})}
               = \frac{\beta_{t,l} \ps{x_t}{z_t=l} A_{kl} \alpha_{t-1,k} }{p^*(x_{1:T})}
\end{align}
and exploit the following recursive relationships:
\begin{align}
  \begin{split}
    \alpha_{t,k} &= p^*(x_{1:t}, z_t) \\
                  &= \sum_l \ps{x_t}{z_t=k} \ps{z_t=k}{z_{t-1}=l} p^*(x_{1:t-1}, z_{t-1}=l) \\ 
                  &= \sum_l \ps{x_t}{z_t=k} A^*_{lk} \alpha_{t-1,l}
  \end{split}\\ 
  \begin{split}
    \beta_{t,k} &= \ps{x_{t+1:T}}{z_t} \\
                &= \sum_l  \ps{x_{t+2:T}}{z_{t+1}=l} \ps{x_{t+1}}{z_{t+1}=l} \ps{z_{t+1}=l}{z_{t}=k}\\ 
                &= \sum_l  \beta_{t+1,l} \ps{x_{t+1}}{z_{t+1}=l} A^*_{kl} 
  \end{split} 
\end{align}
We can now loop \emph{forward} in time to recursively calculate $\alpha_t$ from $\alpha_{t-1}$ and backward in time to calculate $\beta_t$ from $\beta_{t+1}$. The boundary conditions on these passes are:
\begin{align}
  \alpha_{1,k} &= p^*(x_1, z_1) = \ps{x_1}{z_1} p^*(z_1) =  \prod_k \ps{x_1}{z_1=k} \pi^*_k \\
  \beta_{T,k} &= 1
\end{align}
In practice, it proves more convenient to calculate a normalized version of $\hat{\alpha}$ and $\hat{\beta}$. To do so, we introduce a set of scaling factors $c_t$:
\begin{align}
  \begin{split}
    c_t &= \ps{x_t}{x_{1:t-1}}
  \end{split} 
\end{align}
such that normalized forward and backward variables can be defined as:
\begin{align}
  \begin{split}
    \hat{\alpha}_{t,k} &= \frac{\alpha_{t,k}}{p^*(x_{1:t})} 
                        = \prod_{t'=1}^t \frac{1}{c_{t'}} \alpha_{t,k}  \\
    \hat{\beta}_{t,k} &= \frac{\beta_{t,k}}{\ps{x_{t+1:T}}{x_{1:t}}} 
                        = \prod_{t'=t+1}^T \frac{1}{c_{t'}} \beta_{t,k} 
  \end{split}
\end{align}
This choice of normalization implies:
\begin{align}
  \begin{split}
    \gamma_{t,k}
    &=
    \frac{\alpha_{t,k}\beta_{t,k}}{p^*(x_{1:T})} 
    = \frac{\alpha_{t,k}\beta_{t,k}}{\ps{x_{t+1:T}}{x_{1:t}} p^*(x_{1:t})} 
    = \hat{\alpha}_{t,k} \hat{\beta}_{t,k} 
  \end{split}\\
  \begin{split}
    \xi_{t,k,l}
      &= \frac{\beta_{t,l} \ps{x_t}{z_t=l} A_{kl} \alpha_{t-1,k} }{p^*(x_{1:T})} 
      = \frac{c_t \hat{\beta}_{t,l} \ps{x_t}{z_t=l} A_{kl} \hat{\alpha}_{t-1,k} }{p^*(x_{1:T})}
  \end{split}
 \end{align}
The following recursion relations hold for $\hat{\alpha}$ and $\hat{\beta}$:
\begin{align}
  \begin{split}
    c_t \hat{\alpha}_{t,k}  
    &= \sum_l \ps{x_t}{z_t=k} A^*_{lk} \hat{\alpha}_{t-1,l}
  \end{split}\\ 
  \begin{split}
    c_{t+1} \beta_{t,k} 
    &= \sum_l \hat{\beta}_{t+1,l} \ps{x_{t+1}}{z_{t+1}=l} A^*_{kl} 
  \end{split} 
\end{align}
We can now solve for $c_t$ from the recursion relation for $\hat{\alpha}$ using that $\sum_k \hat{\alpha}_{t,k} = 1$:
\begin{align}
  \begin{split}
    c_t 
    &=
    c_t
    \sum_k
    \hat{\alpha}_{t,k}  
    = 
    \sum_{k,l} \ps{x_t}{z_t=k} A^*_{lk} \alpha_{t-1,l}
  \end{split} 
\end{align}
So the scale factors $c_t$ are nothing but the normalization constant for $\hat{\alpha}_t$ and can therefore essentially be obtained for free during the forward pass. Note that these also give us an estimate for $p^*(x)$:
\begin{align}
   p^*(x) &= p^*(x_{1:t}) = \prod_t c_t
\end{align}
which gives us the normalization constant for $q(z)$
\begin{align}
   \hat{Z}_{q(z)} &= \ln p^*(x) = \sum_t \ln c_t
\end{align}


\section{Calculation of the Evidence}

The last thing that remains is to calculate the lower bound so we can check for convergence.
\begin{align}
  \begin{split}
  \L[q(z), q(\theta)]
    &=
    \sum 
    \int \d{\theta }
    \sum_{z }  
    q(z ) q(\theta )
    \ln
    \left[
      \frac{p(x , z , \theta )}
            {q(z ) q(\theta )}  
    \right] \\  
  \end{split}
\end{align}
We can decompose the terms in this equation as:
\begin{align}
  \begin{split}
  \L[q(z), q(\theta)]
    =
    \sum &
    E_{q(z )q(\theta )}
    \left[
      \ln \p{x , z }{\theta }
    \right] \\
    & + 
    D_{KL}[q(\theta ) | p(\theta )]
    -
    E_{q(z )}
    \left[
      \ln q(z )
    \right]  
  \end{split}
\end{align}
Now note from equation (\ref{eq:q_z_star}) that $E_{q(z )} \left[ \ln q(z ) \right]$ can   
be written as:
\begin{align}
  E_{q(z )} \left[ \ln q(z ) \right]
  &=
  E_{q(z )q(\theta )}[\ln \p{x , z }{\theta }] - \ln \hat{Z}_{q(z )}
\end{align}
So
\begin{align}
  \begin{split}
  \L[q(z), q(\theta)]
    = &
    \sum 
    \ln \hat{Z}_{q(z )} + D_{KL}[q(\theta ) \,||\, p(\theta )]
  \end{split}
\end{align}
The term $\ln \hat{Z}_{q(z )}$ is obtained from the forward backward algorithm. The Kullback-Leibler divergence between $q(\theta)$ and $p(\theta)$ decomposes into: 
\begin{align}
  \begin{split}
    D_{KL}[q(\theta ) \,||\, p(\theta )]
    =
    \sum_k & D_{KL}[q(\mu_k,\lambda_k) \,||\, p(\mu_k,\lambda_k)] \\
    & +
    D_{KL}[q(A) \,||\, p(A)]
    +
    D_{KL}[q(\pi) \,||\, p(\pi)]
  \end{split}
\end{align}
The expression for the $D_{KL}$ of a Gaussian-Wishart distribution is a bit painful, but can be obtained from Bishop equations (10.74) and (10.77).
\begin{align}
    D_{KL}[q(\mu_k,\lambda_k) \,||\, p(\mu_k,\lambda_k)]
    = & E_{q(\mu_k,\lambda_k)}[p(\mu_k,\lambda_k)]
        - E_{q(\mu_k,\lambda_k)}[q(\mu_k,\lambda_k)] 
\end{align}
which expands to
\begin{align}
  \begin{split}
    E_{q(\mu_k,\lambda_k)}[p(\mu_k,\lambda_k)]
    = &
    \frac{1}{2}
    \left[ 
      \ln \left( \frac{u^\beta_k}{2\pi} \right)
      +
      \ln \lambda^*_k
      - 
      \frac{u^\beta_k}{w^\beta_k} 
      +
      u^\beta_k u^\nu_k w^W_k (w^\mu_k - u^\mu_k)^2
    \right]
  \end{split}\\
  \begin{split}
    E_{q(\mu_k,\lambda_k)}[q(\mu_k,\lambda_k)]
    = &
    \frac{1}{2}
    \left[ 
      \ln \left( \frac{u^\beta_k}{2\pi} \right)
      +
      \ln \lambda^*_k
      - 
      1
      -
      H[\lambda_k]
    \right]
  \end{split}
\end{align}
with
\begin{align}
  H[\lambda_k]
  &= - \ln \left[(2/u^W_k)^{u^\nu_k / 2} \Gamma(w^\nu_k/2)\right] - \frac{w^\nu_k - 2}{2} \ln w^\nu_k w^W_k + \frac{w^\nu}{2}
\end{align}
The KL divergences for $A$ and $\pi$ have simple closed-form expressions:
\begin{align}
  D_{KL}[q(A_k) \,||\, p(A_k)]
  =& \sum_l [w^A_{k,l} - u^a_{k,l}] [\psi(w^A_{k,l}) - \psi(u^A_{k,l})]\\
  D_{KL}[q(\pi) \,||\, p(\pi)]
  =& \sum_l [w^\pi_l - u^\pi_l] [\psi(w^\pi_l) - \psi(u^\pi_l)]
\end{align}

\chapter{Hierarchical Updates (Empirical Bayes)}

In the hierarchical step we maximize the summed lower bound log-evidence with respect to the ensemble distribution $\p{\theta}{u}$. This step can be understood as a type of Emperical Bayes method. 

In the more general case of Empirical Bayes, one would introduce a prior $p(u)$, and run alternating variational updates to find (approximations) of the posteriors $\p{\theta}{x}$ and $\p{u}{x}$:
\begin{align}
  \p{z,\theta}{x}
  &=
  \frac{\p{x}{z,\theta}}{p(x)} \int \d u \p{z}{\theta}\p{\theta}{u} p(u) \\
  \p{u}{x}
  &=
  \frac{p(u)}{p(x)} \sum_z \int \d \theta \p{x}{z, \theta} \p{z}{\theta} \p{\theta}{u}
\end{align}
Of course, calculation of the hierarchical generalization of the evidence $p(x)$ would require an additional integral:
\begin{align}
  p(x)
  &=
  \int \d u \p{x}{u} p(u)
\end{align}
One could now in principle attempt to construct a variational approach in terms of 3 distributions $q(z), q(\theta), q(u)$, that minimizes a lower bound on the log hierarchical evidence $\log p(x)$. However, this would be a lot of pain, for not so much gain.

A simpler approach is to construct an EM algorithm to obtain a point estimate for $u$. The quantity optimized is the summed lower bound log evidence over the ensemble:
\begin{align}
  \log \p{x}{u} &\simeq \sum_n \L_n
\end{align}
The E-step amounts to running VBEM on every trace to construct:
\begin{align}
 \q{\theta}{w} &= \prod_n \q{\theta}{w_n} \simeq \p{\theta}{x, u}
\end{align}
Whereas the M-step maximizes the summed lower bound w.r.t. $u$:
\begin{align}
  0 &= \pdff{}{u} \sum_n \L_n \\
    &= \pdff{}{u} \sum_n \int \d{\theta_n} \q{\theta_n}{w_n} \log \p{\theta_n}{u} \\
    &= \sum_n \int \d{\theta_n} 
        \q{\theta_n}{w_n} \frac{\partial_u \p{\theta_n}{u}}{\p{\theta_n}{u}}
       \label{eq:u_maximization}
\end{align}
Now note that $p(\theta)$ factorizes without need for further approximation
\begin{align}
 \p{\theta}{u} = \p{\mu,\lambda}{u^\mu, u^\beta, u^W, u^\nu} \p{A}{u^A} \p{\pi}{u^\pi}
\end{align}
so the updates for each factor can be computed separatedly. 

\section{Conjugate-Exponential Form}

If we rewrite $\p{\theta}{u}$ to its conjugate exponential form $\p{\eta}{\nu,\chi}$, the expression in equation \ref{eq:u_maximization} takes the form:
\begin{align}
    0
    &= 
    \sum_n \int \d{\eta_n} 
        \q{\eta_n}{\nu_n, \chi_n} 
        \frac{\partial_{\nu,\chi} \p{\eta_n}{\nu, \chi}}
             {\p{\eta_n}{\nu,\chi}}
       \label{eq:u_max_ce}
\end{align}
Here we adopt the convention where $\{\nu,\chi\}$ are taken to signify the hyperparameters of the ensemble distribution, whereas $\{\nu_n, \chi_n\}$ denotes the set of variational parameters for the approximate posterior of each trace. 

The derivatives of $\p{\eta}{\nu,\chi}$ with respect to the hyperparameters are given by:
\begin{align}
    \pdff{\p{\eta}{\nu,\chi}}{\nu}
    &=
    \left[
      \frac{\partial_\nu h(\nu,\chi)}{h(\nu,\chi)}
      +
      \ln g(\eta) 
    \right]
    \p{\eta}{\nu,\chi}
    \\
    \nabla_\chi \p{\eta}{\nu,\chi}
    &=
    \left[
      \frac{\nabla_\chi h(\nu,\chi)}{h(\nu,\chi)}
      +
      \eta 
    \right]
    \p{\eta}{\nu,\chi}
\end{align}
If we now substitute these expressions in equation \ref{eq:u_max_ce}, we obtain the expressions:
\begin{align}
  0
  &=
  \pdff{}{\nu} \sum_n \L_n
  = 
  \sum_n 
   E_{q(\eta_n)}
   \left[
      \frac{\partial_\nu h(\nu,\chi)}{h(\nu,\chi)}
      +
      \ln g(\eta) 
   \right]
   = 
\end{align}
Given that terms containing $h(\nu,\chi)$ have no dependence on $\eta$ we can rewrite these equalities as:
\begin{align}
   E_{q(\eta)}
   \left[
     \ln g(\eta)
   \right]
   &=
   \frac{1}{N}
   \sum_n
   E_{q(\eta_n)}
   \left[
     \ln g(\eta)
   \right] \\
   &=
   -\frac{\partial_\nu h(\nu,\chi)}{h(\nu,\chi)}
   \\
   E_{q(\eta)}
   \left[
     \eta
   \right]
   &=
   \frac{1}{N}
   \sum_n
   E_{q(\eta_n)}
   \left[
     \eta
   \right] \\
   &=
   -\frac{\nabla_\chi h(\nu,\chi)}{h(\nu,\chi)}
\end{align}
These equations implicitly specify the update conditions for the hyperparameters in terms of the averaged expectation values of $\eta$ and $\ln g(\eta)$ under the approximate posterior for each trace.

The expectation values for $\ln g(\eta)$ and $\eta$ can be computed by noting that the integral of a probability density function must always equal to 1, implying that it's derivatives w.r.t. $\nu$ and $\chi$ must be zero:
\begin{align}
  0
  &=
  \pdff{}{\nu_n}
  \int \d{\eta_n}
  \q{\eta_n}{\nu_n, \chi_n}
  =
  \frac{\partial_{\nu_n} h(\nu_n, \chi_n)}
       {h(\nu_n, \chi_n)}
  +
  E_{q(\theta_n)} [ \ln g(\eta_n) ] 
  \\
  0
  &=
  \nabla_{\chi_n}
  \int \d{\eta_n}
  \q{\eta_n}{\nu_n, \chi_n}
  =
  \frac{\nabla_{\chi_n} h(\nu_n, \chi_n)}
       {h(\nu_n, \chi_n)}
  +
  E_{q(\theta_n)} [\eta_n] 
\end{align}
So the logarithmic derivates of $h(\nu, \chi)$ in fact gives us the required expectation values, and the equations for the hyperparameter updates are in fact equivalent to the expressions:
\begin{align}
   E_{p(\eta)}
   \left[
     \ln g(\eta)
   \right]
   &=
   E_{q(\eta)}
   \left[
     \ln g(\eta)
   \right]
   \\
   E_{p(\eta)}
   \left[
     \eta
   \right]
   &=
   E_{q(\eta)}
   \left[
     \eta
   \right]
\end{align}
or
\begin{align}
   \frac{\partial_\nu h(\nu,\chi)}{h(\nu,\chi)}
   &=
   \frac{1}{N}
   \sum_n
   \frac{\partial_{\nu_n} h(\nu_n,\chi_n)}{h(\nu_n,\chi_n)}
   \\ 
   \frac{\nabla_\chi h(\nu,\chi)}{h(\nu,\chi)}
   &= 
   \frac{1}{N}
   \sum_n
   \frac{\nabla_{\chi_n} h(\nu_n,\chi_n)}{h(\nu_n,\chi_n)}
\end{align}

\section{Emission Distribution (Normal-Wishart)}

For a 1-dimensional Normal-Wishart distribution the conjugate-exponential representation (section \ref{ssec:conj-exp-norm-wish}) takes the form:
\begin{align}
  \eta &= \{ \lambda, \lambda \mu \} \\
  \nu &= u^\beta = u^\nu - 1 \\
  \chi &= \{ -\tfrac{1}{2}(u^\beta (u^\mu)^2 + 1/u^W), u^\beta u^\mu \} \\
  g(\eta) &= (\eta_1 / 2\pi)^{1/2} \exp [-\eta_2^2 / 2 \eta_1] \\
  h(\nu, \chi) &= (2 \pi)^{(\nu-1)/2}  \nu^{1/2}  (-\chi_1 - \chi_2^2/\nu)^{(\nu+1)/2}
\end{align}
The expressions for the expectation values of $\ln g$ and $\eta$ become:
\begin{align}
  E_{q(\theta_n)} [ \ln g ] 
  &=
  - \frac{1}{2} [ 1 / w_n^{\beta} +  w_n^\nu w_n^W (w_n^\mu)^2
                   + \log(\pi / w_n^W) - \psi(w_n^\nu/2) ] \\
  E_{q(\theta_n)} [ \lambda ]
  &=
  w_n^\nu w_n^W \\
  E_{q(\theta_n)} [ \lambda \mu ]
  &=
  w_n^\nu w_n^W w_n^\mu
\end{align}
We now obtain the following updates for $u^\mu$ and $u^W$:
\begin{align}
  u^\mu &= E_{q(\theta)} [ \lambda \mu ] / E_{q(\theta)} [ \lambda ] \\
  u^W &= E_{q(\theta)} [ \lambda ] / u^\nu \\
\end{align}
And an implicit expression for for $\nu = u^\beta = u^\nu -1$ that must be solved numerically:
\begin{align}
    - \frac{1}{2} 
    \left[ 
      \frac{1}{u^\nu -1}
      +  \frac{E_{q(\theta)} [ \lambda \mu ]^2}{E_{q(\theta)} [ \lambda ]}
      + \log \left(
               \frac{\pi u^\nu}{E_{q(\theta)} [ \lambda ]}
             \right)
      - \psi(u^\nu/2) 
    \right] 
    &=
    \frac{1}{N} \sum_n E_{q(\theta_n)} [\ln g] 
\end{align}

\section{Inital State and Transition Probabilities (Dirichlet)}

For a Dirichlet distribution the conjugate exponential forms (section \ref{ssec:conj-exp-dir}) are given by:
\begin{align}
  \eta &= \{ \ln \pi_k \} \\
  \chi &= \{ u^\pi_k \} \\
  h(\chi) &= \frac{\prod_k \Gamma(\chi_k+1)}{\Gamma(\sum_k (\chi_k + 1))}
\end{align}
And the log expectation value of $\eta$ is:
\begin{align}
 E_{q(\theta_n)} [ \eta ] &= E_{q(\theta_n)} [ \ln \pi ] = \psi(w^\pi_{n,k}) - \psi(\sum_k w^\pi_{n,k})
\end{align}
which again leads to a coupled set of implicit equations that must be solved numerically: 
\begin{align}
  \psi(u^\pi_{k}) - \psi(\sum_k u^\pi_{k}) 
  &= 
  \frac{1}{N} \sum_n
  \psi(w^\pi_{n,k}) - \psi(\sum_k w^\pi_{n,k})
\end{align}
The updates for each row of the transition matrix are performed in the same manner   
\begin{align}
  \psi(u^A_{kl}) - \psi(\sum_l u^A_{kl}) 
  &= 
  \frac{1}{N} \sum_n
  \psi(w^A_{n,kl}) - \psi(\sum_l w^A_{n,kl})
\end{align}

\section{Mixtures of Priors}

The empirical Bayes approach admits a straightforward generalization to inference over mixtures of ensemble distributions. Suppose that a latent state $y = {1 \ldots M}$ encodes the membership of each trace with respect to a set of M sub-populations in the ensemble, which have different parameter distributions $\p{\theta | u_m}$. The evidence can now be expressed as a marginal over $y$:
\begin{align}
 \p{x}{u} 
 &= \sum_y \p{x}{u,y} p(y) \\
 &= \sum_m \p{x}{u_m} p(y=m) \\
 &\geq \sum_{n,m} \exp(\L_{nm}) p(y=m) 
\end{align}
where $\L_{nm} \geq \log \p{x_n}{u_m}$ is the lower bound log evidence for trace $n$ with respect to mixture component $m$.

An expectation maximization algorithm over this mixture can be now be constructed by introducing a variational posterior $q(y_n \!=\! m) = \omega_{nm}$ for each trace. The corresponding (approximate) E-step is now given by:
\begin{align}
  q^{(i+1)}(y_n \!=\! m) 
  &=
  \frac{\exp(\L_{nm}) p^{(i)}(y \!=\! m)}{\sum_l \exp(\L_{nl}) p^{(i)}(y \!=\! l)}        
  = \omega^{(i+1)}_{nm}
\end{align}
And the M-step simply becomes:
\begin{align}
  p^{(i+1)}(y = m)
  &=
  \frac{1}{N} \sum_n q^{(i+1)}(y_n = m)
\end{align}
The mixed version of the hierarchical algorithm now maximizes the lower bound
\begin{align}
 \log \p{x}{u}
 &\geq
 \sum_{n} \sum_{y_n} q(y_n) \log \left[ \frac{\p{x_n}{u, y_n}}{q(y_n)} \right]\\
 &\geq
 \sum_{n,m} \omega_{nm} \left[ \L_{nm} -  \log \omega_{nm} \right] \\
 &=
 \L
\end{align}
and the hierarchical update for the $m$-th subpopulation becomes equivalent solving of the equations  
\begin{align}
 0
 &=
 \sum_{n,m} \omega_{nm} \pdff{\L_{nm}}{u_m}
\end{align}
which produces a set of update equations analogous to the single-population case, where the expectation values with respect to the approximate posteriors are now weighted by $\omega$:
\begin{align}
 E_{\p{\theta}{u_m}} [ \ln g ]
 &= 
 \frac{1}{\sum_n \omega_{nm}} \sum_n \omega_{nm} E_{\q{\theta_n}{w_{nm}}} [ \ln g ] \\
 E_{\p{\theta}{u_m}} [ \eta ]
 &= 
 \frac{1}{\sum_n \omega_{nm}} \sum_n \omega_{nm} E_{\q{\theta_n}{w_{nm}}} [ \eta ]
\end{align}


\end{document}
